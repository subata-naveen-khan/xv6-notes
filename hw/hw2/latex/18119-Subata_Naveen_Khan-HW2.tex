\documentclass{hw}

\title{Operating Systems - Homework 2}
\author{Subata Naveen Khan - 18119}
\date{3rd October 2024}

\begin{document}
\maketitle

\begin{enumerate}[itemsep=.4cm]
\item % q1
\texttt{ecall} is a RISC-V instruction that raises the hardware privilege level from user mode to supervisor mode. A user program in xv6 the \texttt{ecall} instruction to call a system call. \texttt{ecall} triggers the CPU to jump to the supervisor trap handler in \texttt{trap.c}. The trap handler checks the \texttt{a7} register for the system call number and handles it accordingly from the kernel.

\item % q2
The main reason this code cannot be implemented in xv6 is that xv6 is a bare bones operating system with limited functionality, so it does not support the operations required: 
\begin{itemize}
    \item xv6 does have an \texttt{ls} command, but it doesn't have higher-level functionality such as \texttt{-al}.
    \item xv6 does not have a built-in \texttt{tr} command. 
    \item The xv6 implementations of some system calls such as \texttt{fork()} and \texttt{wait()} are very limited in comparison to standard Unix systems. 
    \item Other functions like \texttt{perror()} and \texttt{execvp()} are also not available in xv6 and would require modifications to the code to implement them.
\end{itemize}
% \begin{lstlisting}
%     hello       
% \end{lstlisting}

\item   % QUESTION 3
    \begin{itemize}
    \item Store the table in the kernel memory with no reference to it in user space.
    \item Mark the pages storing the table data to not give user space the permission to read or write to them.
    \item Kernel Address Space Layout Randomization (KASLR): Randomize the location of the table every time the system boots.
    \item Prevent writing access to the table completely after initialization is complete. 
    \end{itemize}

\item   % QUESTION 4
    \begin{enumerate}[label=\alph*.]
    \item \texttt{spinlock}: used to disable interrupts and implement mutual exclusion in multi-core environments.
    \item \texttt{proc *parent}: points to the parent process in order to maintain the process hierarchy.
    % \item \texttt{pagetable}: used to map virtual addresses that the process interacts with, to the physical addresses in the main memory.
    \item \texttt{trapframe}: used to store the CPU's registers that contain the process's state when a trap occurs, to be restored after the kernel deals with the trap.
    \item \texttt{context}: used to store the CPU's registers that contain the process's state when context switching to another process, to be restored when the kernel resumes execution.
    \item \texttt{file *ofile[]}: used to store pointers to all the open file descriptors that the process has opened
    \item \texttt{inode *cwd}: used to keep track of the process's location.
    \end{enumerate}
    
\item   % QUESTION 5
    \begin{itemize}
    \item Implement KASLR.
    \item Ensure that system calls cannot be made directly from the user space.
    \item Check that arguments passed from the user space point to regions in the memory that the user space is allowed access to.
    \end{itemize}

\item The \texttt{t} registers are temporary, so they are not preserved when a trap or context switch occurs. So the system call number cannot be placed there. 

The \texttt{s} registers are saved with the rest of the state of the process, so they contain relevant to the process, which the system call number is not.

The \texttt{a} registers are used to pass information between functions. So it makes sense for the system call number, which is an argument used by the \texttt{syscall()} function, to be stored in them.

\item 
The CLINT handles local software and timer interrupts that are specific to the core. It is used for inter-process communication and time-sharing processes.

The PLIC handles hardware interrupts from external devices such as a disk, keyboard, or network interface. It determines the priority of each interrupt and is responsible for routing them to the appropriate cores. 

\item When a system call is made, \texttt{usys.S} places the relevant system call number (defined in \texttt{kernel/syscall.h}) into the \texttt{a7} register. Due to the hacking, the number stored in \texttt{a7} will not correlate with the values in \texttt{syscall.h}. Then \texttt{ecall} will be executed. In the kernel, the trap handler will check the cause of the trap and accordingly call the \texttt{syscall()} function in the \texttt{kernel/syscall.c} file. The value in \texttt{a7} will be stored in the \texttt{num} variable. In the \texttt{if} statement, the \texttt{\textbf{num < NELEM(syscalls)}} condition will return false, so the intended system call will not be executed and the \texttt{else} block will run instead. 

\end{enumerate}

\end{document}